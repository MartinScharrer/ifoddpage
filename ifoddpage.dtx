% \iffalse meta-comment
%<=*COPYRIGHT>
%% Copyright (C) 2011-2012 by Martin Scharrer <martin@scharrer-online.de>
%% ----------------------------------------------------------------------
%% This work may be distributed and/or modified under the
%% conditions of the LaTeX Project Public License, either version 1.3
%% of this license or (at your option) any later version.
%% The latest version of this license is in
%%   http://www.latex-project.org/lppl.txt
%% and version 1.3 or later is part of all distributions of LaTeX
%% version 2005/12/01 or later.
%%
%% This work has the LPPL maintenance status `maintained'.
%%
%% The Current Maintainer of this work is Martin Scharrer.
%%
%% This work consists of the files ifoddpage.dtx and ifoddpage.ins
%% and the derived filebase ifoddpage.sty.
%%
%<=/COPYRIGHT>
% \fi
%
% \iffalse
%<*driver>
\ProvidesFile{ifoddpage.dtx}[%
%<=*DATE>
    2016/04/23
%<=/DATE>
%<=*VERSION>
    v1.1
%<=/VERSION>
    DTX file for ifoddpage package]
\documentclass{ydoc}
\GetFileInfo{ifoddpage.dtx}
\usepackage{ifoddpage}[\filedate]
\EnableCrossrefs
\CodelineIndex
\RecordChanges
\begin{document}
  \DocInput{ifoddpage.dtx}
  \PrintChanges
  \PrintIndex
\end{document}
%</driver>
% \fi
%
% \CheckSum{90}
%
% \CharacterTable
%  {Upper-case    \A\B\C\D\E\F\G\H\I\J\K\L\M\N\O\P\Q\R\S\T\U\V\W\X\Y\Z
%   Lower-case    \a\b\c\d\e\f\g\h\i\j\k\l\m\n\o\p\q\r\s\t\u\v\w\x\y\z
%   Digits        \0\1\2\3\4\5\6\7\8\9
%   Exclamation   \!     Double quote  \"     Hash (number) \#
%   Dollar        \$     Percent       \%     Ampersand     \&
%   Acute accent  \'     Left paren    \(     Right paren   \)
%   Asterisk      \*     Plus          \+     Comma         \,
%   Minus         \-     Point         \.     Solidus       \/
%   Colon         \:     Semicolon     \;     Less than     \<
%   Equals        \=     Greater than  \>     Question mark \?
%   Commercial at \@     Left bracket  \[     Backslash     \\
%   Right bracket \]     Circumflex    \^     Underscore    \_
%   Grave accent  \`     Left brace    \{     Vertical bar  \|
%   Right brace   \}     Tilde         \~}
%
%
% \changes{v1.0}{2011/09/13}{First version.}
% \changes{v1.1}{2016/04/23}{Fixed misspelled macro name.}
%
%
% \DoNotIndex{\newcommand,\newenvironment}
%
% \GetFileInfo{ifoddpage.dtx}
% \author{Martin Scharrer}
% \email{martin@scharrer.me}
% \repository{https://bitbucket.org/martin_scharrer/ifoddpage}
% \maketitle
%
% \section{Introduction}
% Sometimes it is required to know if a macro is placed on an odd or even page.
% Simply testing the value of the \texttt{page} counter is not reliable, because \TeX\ reads and processes often more material than fits on the currently processed
% page in order to decide where to place the page break. Also moving material like floats can be placed on completely different pages than where the surrounding material
% is placed. The only reliable way to determine the correct page number is to write it to the auxiliary file using a label and read it back in the next compiler run.
% Labels use a so called \emph{whatsit}s (see \emph{The TeXBook}) to write the given code to the auxiliary file \emph{just when the material is shipped out}, i.e.\ a full page has 
% been filled and is written to the output file. This way ensures that the determined page number is always correct, but adds an overhead.
%
% This package implements the described technique. A macro must be used to generate and read-back the label before the conditional can be used.
% A counter is used internally to provide unique names to the labels.
%
%
% \section{Usage}
% The following macros are provided for \LaTeX\ document authors:
%
% \DescribeMacro{\checkoddpage}
% This macro checks if the current page is odd or even by placing a label to the auxiliary file and reading it back in.
% This requires a second compiler run to work perfectly, but the code falls back to use the \texttt{page} counter value 
% if the label is not yet defined, i.e.\ the label was just added or no auxiliary file existed.
% This fallback code might return wrong values for material just around page breaks, but
% is better than using a constant.
%
% Note that this macro inserts an invisible element (a so called \emph{whatsit}) which is placed on the page.
% It is possible that this happens to be at the very end of the last line of the page and any other code following it would then be placed
% on the next page. Some code like |\checkoddpage \ifoddpage odd\else even\fi| might then place the first macro on the current, say even, page but
% print `|even|' at the top of the next page which is odd. In order to avoid this both the check macro and the output text can be placed in the same
% box (e.g.\ |\mbox|). If the full content can't be placed in a box, e.g.\ because line breaking is required a |\mbox| can also be substituded by 
% |\leavevmode| (to start a paragraph if required) and |\hbox| where the |{}| are replaced by |\bgroup| and |\egroup| so two alternative box ends can be specified:
%
% \begin{lstlisting}[gobble=2,numbers=none,basicstyle=\ttfamily]
%    \leavevmode\hbox\bgroup
%    \checkoddpage
%    \ifoddpage
%      odd\egroup\ more text
%    \else
%      even\egroup\ other text
%    \fi
% \end{lstlisting}
%
% \noindent
% This boxes the \Macro\checkoddpage with the first word of both clauses. Note that in this case it the settings are local to the box scope and \Macro\ifoddpage should
% be used outside of it again (without another \Macro\checkoddpage).
%
% \DescribeMacro{ifoddpage}
% This \TeX\ conditional is set (locally) by \Macro\checkoddpage.
% It is true if the current page is odd, or false if it is even. This is independent from whether the \texttt{oneside} or \texttt{twoside} mode is used.
%
% \DescribeMacro{ifoddpageoroneside}
% This \TeX\ conditional is set (locally) by \Macro\checkoddpage.
% Like the previous conditional it is true if the current page is odd, or false if it is even.
% However, if the \texttt{oneside} mode is active it is always true.
% This is useful for code which needs to check if the odd or even page layout is used, because in \texttt{oneside} mode the odd page layout is used for all pages.
%
% \par\medskip\par\noindent
% The following macros are intended for package authors:
%
% \DescribeMacro{\@ifoddpage}{<true>}{<false>}
% This \LaTeX\ macro uses \Macro{ifoddpage} and executes its first argument if that conditional is true but the second argument if it is false.
% The \Macro\checkoddpage must be used closely beforehand to get correct results.
%
% \DescribeMacro{\@ifoddpageoroneside}{<true>}{<false>}
% This \LaTeX\ macro uses \Macro{ifoddpageoroneside} and executes its first argument if that conditional is true but the second argument if it is false.
% The \Macro\checkoddpage must be used closely beforehand to get correct results.
%
% \DescribeMacro{\oddpage@page}
% This macro expands (using multiple steps) to a text representation of the page number for the last \Macro\checkoddpage.
% It is used in this macro together with \Macro\ifodd to set the provided conditionals.
%
%
% \section{Similar packages}
% The \pkg{changepage} package and the \pkg{memoir} class also provide the core functionality of this package, i.e.\ they define \Macro\checkoddpage and \Macro\ifoddpage but no other of the mentioned macros.
% That package uses this conditional internally in order to allow the user to change the page layout for odd or even pages, which is its main function.
% The \pkg{ifoddpage} should work together with that package and class, but should be loaded last to ensure the correct functionality of \Macro{ifoddpageoroneside}.
% It also uses labels if it is loaded using the \texttt{strict} option.
% The \pkg{ifoddpage} has the following benefits over \pkg{changepage} (if the page layout doesn't have to be changed):
%
% \begin{itemize}
%   \item Smaller code base, because no additional functionality is provided.
%   \item The \Macro{ifoddpageoroneside} conditional is also provided which simplifies code which has to handle odd/even margins.
%   \item The label code uses the same internal \LaTeX-core code as the normal \Macro\label and generates identical error, warning and info messages.
%   \item The fallback page number is not constant (0) but the current \texttt{page} counter value is used, which is a good approximation.
% \end{itemize}
%
% \StopEventually{}
%
% \clearpage
% \section{Implementation}
%
% \iffalse
%<@ifoddpage.sty>
% \fi
%
% \Finale
% \endinput
